\documentclass[a4paper]{report}
\pagestyle{headings}
\usepackage{hyperref}
\usepackage{listings}
\usepackage{graphicx}
\usepackage{subfiles}
\usepackage{multirow}
\usepackage[table,xcdraw]{xcolor}
\lstset{numbers=right}
\lstset{breaklines}
\title{Lab Report for Software Engineering course \newline
 Lab 5: Demand Change and Prototype Development}
\author{Wang, Chen\qquad Liu, Jiaxing\qquad Huang, Jiani\qquad Tang, Xinyue \\
16307110064\qquad17302010049\qquad 17302010063\qquad 16307110476 \\
School of Software\\
Fudan University
}
\date{\today}
\bibliographystyle{plain}
\begin{document}
\maketitle

\tableofcontents
\chapter{Demands of this lab}
\section{Requirements of this lab}
According to the documentation of the lab assigner, this lab should satisfy the requirments in the following perspectives:
\subsection{Documentation}
In the documentation, we need to accomplish two parts, that is, we need to:
Understand the needs of this experiment, complete the requirements document;
Organize overall design documentation and detailed design documentation based on requirements.
\subsection{Code Style}
Before the experiment, our group members should agree on the code specification and the unified code style.
\subsection{Project Management}
The project is generally hosted on the Huawei Devcloud platform, on which platform we are going to finish the following tasks:
Project management should be arranged based on the DevCloud platform, including adding work items, assigning work items, associating work items, managing work item status, and other project management functions;
\par
Issues in the development process need to be recorded and managed on DevCloud, including defect reports in integrated development;
\par
Collaborative development based on Git, submit project documentation and code on a team basis.
\section{Specifications of the Lab}

\chapter{Division of work for this lab}



\chapter{Analysis of the demands}



\chapter{General design for the implementation}



\chapter{Detailed design for the implementation}




\chapter{Problems encountered in this project}


\chapter{Measures against demand change}



\chapter{Tools and literature involved in this project}



\chapter{Conclusion for the process of accomplishing this project}


\begin{thebibliography}{A}


\bibitem{1}
Wikipedia contributors. (2019, March 22). JUnit. In \emph{Wikipedia, The Free Encyclopedia}. Retrieved 14:53, April 1, 2019, from \url{https://en.wikipedia.org/w/index.php?title=JUnit&oldid=888928403}

\end{thebibliography}
\end{document} 